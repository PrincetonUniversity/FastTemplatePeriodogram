\documentclass[notitlepage]{article}
\usepackage{amssymb}
\usepackage{amsmath}
\usepackage[margin=0.5in]{geometry}
\title{Fast template periodogram}
\author{John Hoffman}


\newcommand{\todo}[1]{{\bf #1}}
\newcommand{\Order}[1]{\mathcal{O}\left(#1\right)}
\newcommand{\Cc}[1][n]{\widetilde{C}}
\newcommand{\Ss}[1][n]{\widetilde{S}}
\newcommand{\CCc}[1][nm]{\widetilde{CC}}
\newcommand{\SSs}[1][nm]{\widetilde{SS}}
\newcommand{\CS}[1][nm]{\widetilde{CS}}
\newcommand{\YC}[1][n]{\widetilde{YC}}
\newcommand{\YS}[1][n]{\widetilde{YS}}
\newcommand{\Xs}{X_s}
\newcommand{\Xc}{X_c}
\newcommand{\Xst}{X_{s, \tau_0}}
\newcommand{\Xct}{X_{c, \tau_0}}
\newcommand{\Ybar}{\bar{y}}
\newcommand{\YY}{YY}
\newcommand{\dM}{\partial_bM}
\newcommand{\hatYC}[1][n]{\widehat{YC}}
\newcommand{\hatYS}[1][n]{\widehat{YS}}
\newcommand{\hatCS}[1][nm]{\widehat{CS}}
\newcommand{\hatCC}[1][nm]{\widehat{CC}}
\newcommand{\hatSS}[1][nm]{\widehat{SS}}
\newcommand{\somt}[1][n]{\sin{{#1}\omega t}}
\newcommand{\comt}[1][m]{\cos{{#1}\omega t}}

\newcommand{\iprod}[2]{\left<#1, #2\right>}
\newcommand{\oprod}[2]{#1 \otimes #2}

\newcommand{\dA}{(\partial_b A)}
\newcommand{\dB}{(\partial_b B)}

\begin{document}
\maketitle

\section{Notes}

Jake, this is an active document that I've been working on. At one point it was the rough draft
of the paper (hence the introduction), but at this point it's more or less a personal log 
during my attempts to derive the fast 
template periodogram. I would ignore everything else except for the math...

I want to know if you have any suggestions/corrections/tips for speeding up this derivation 
process. I am not familiar with Mathematica; would you recommend that I start using
Mathematica or some other program to perform the simplification?

\section{Introduction}

We seek a particular closed-form solution to the periodogram 

\begin{equation}
p(\omega)\equiv \frac{\chi^2 - \chi^2_0}{\chi^2_0}
\end{equation}

where

\begin{equation}
\chi^2 \equiv \sum_i \frac{\left(y_i - y(t_i)\right)^2}{\sigma_i^2}
\end{equation}

for data $y_i$ measured at times $t_i$ with uncertainty $\sigma_i$ with 
respect to a best-fit model $y(t)$. 

The $0$ subscript in $\chi^2_0$ denotes the $\chi^2$ value for a
 model consisting of only a constant factor, i.e., $y(t) = c$ for some $c\in\mathbb{R}$. 

In the canonical case (Lomb-Scargle), $y_{\rm LS}(t) = a \cos\omega t + b\sin\omega t$, which was extended by
\todo{citation for general lsp} to include a constant offset:

\begin{equation}
\hat{y}_{\rm GLS}(t) = a\cos\omega t + b\sin\omega t + c
\end{equation}

This was extended further by \todo{citation for multi-harmonic LSP} to include additional subharmonics
of $\omega$:

\begin{equation}
\hat{y}(t) = c + \sum_{h=1}^{H} \left(a_h\cos h\omega t + b_h\sin h\omega t \right).
\end{equation}

Where $H$ is the number of harmonics to be considered. In this case, there are $2H + 1$ free parameters
to be constrained. While capable of overcoming some of the problems 
typical of the original and generalized versions of the Lomb-Scargle
periodogram (poor ability to constrain non-sinusoidal shapes, etc.), the extra free parameters
reduce the detectability of low amplitude signals in noisy data.

Template periodograms, on the other hand, allow for a more generalized model $\hat{y}(t|\vec{\theta})$, where
$\vec{\theta}$ is the vector of free-parameters for the model. When searching for a known signal-shape
in noisy data, template periodograms do as well as, or better than, LSP/GLSP algorithms, since they
reduce the number of free parameters by constraining the shape of the signal. They are also optimized
for finding signals with shapes similar to that of the template.

However, the disadvantage of template periodograms as they have so far been expressed, 
is that slower and more computationally expensive calculations are needed to produce 
the periodogram compared with the standard LSP/GLSP. The (G)LSP can be improved from an 
$\Order{N^2}$ calculation to an $\Order{N\log N}$ calculation \todo{find nlogn lsp citation}
by utilizing FFT's to accelerate the computation of various sums. A similar speedup can be
gained for \todo{to be continued...}


\section{derivations}

The model $\hat{y}(t)$ associated with the template $M(t)$ is:

\begin{equation}
\hat{y}(t) = aM(t - \tau) + c,
\end{equation}

where $a$, $\tau$, and $c$ are all free parameters with real values. If we express $M(t)$ as a Fourier
series, we have that

\begin{equation}
M(t|\omega) = \sum_{n=1}^H \left[c_n\cos n\omega t + s_n\sin n \omega t\right]
\end{equation}

Where $H$ is $\infty$ in general, but for sufficiently smooth templates, can be
reduced to a finite (and ideally small) natural number, $H\lesssim 10$. 

The Fourier components $c_n$ and $s_n$ are assumed to be real-valued 
and should be normalized in some convenient way.

Furthermore, it is convenient to replace $\tau$ with $b\equiv\cos\omega\tau$. As such, $M(t - \tau)$ becomes

\begin{align}
M(t|\underbrace{a, b, c}_{\vec{\theta}}, \omega) = \sum_n^H &\underbrace{\left[c_nT_n(b) \mp s_nU_{n-1}(b)\sqrt{1 - b^2}\right]}_{g_n(b|c_n, s_n)}\comt[n]\\
					&+\underbrace{\left[s_nT_n(b) \pm c_nU_{n-1}(b)\sqrt{1 - b^2}\right]}_{g_n(b|s_n, -c_n)}\somt[n],
\end{align}

where $T_n(x)$ is the Chebyshev polynomial of the $n$-th kind; recall $\cos nx = T_n(\cos x)$. The presence
of the $\pm$ ambiguity arises because $\sin\omega\tau$ can have either a positive or negative sign, and this
is independent of the value or sign of $b$. The problem is thus split into two cases, and the periodogram value
is $\max(P_+, P_-)$, where $P_+$ is the periodogram value for the case when $\sin\omega\tau \geq 0$ and $P_-$
is the periodogram value for $\sin\omega\tau < 0$.

We can further define
$A_n(b) \equiv g_n(b|c_n, s_n)$ and $B_n(b) \equiv g_n(b|s_n, -c_n)$.

At a given frequency $\omega$, we seek the free parameters $\vec{\theta} = \{a, b, c\}$ that minimize $\chi^2(\vec{\theta})$
for the Fourier-expanded template.

We can use the nature of the optimal solution, namely that 
\begin{equation}
\partial_{\theta_i}\chi^2(\vec{\theta}_{\rm best}) = 0\,\,\,\,\forall \theta_i 
\end{equation}

to generate a system of three equations (with three ``unknowns'': $a$, $b$, $c$):

\begin{eqnarray}
0 &=& \partial_{a} \chi^2 = 2W\sum_i w_i \left(y_i - \hat{y}(t_i | \vec{\theta})\right)M(t_i | \vec{\theta}, \omega)\\
0 &=& \partial_{b} \chi^2 = 2W\sum_i w_i \left(y_i - \hat{y}(t_i | \vec{\theta})\right)\left.\frac{\partial M(t | \vec{\theta}, \omega)}{\partial b} \right|_{t=t_i}\\
0 &=& \partial_{c} \chi^2 = 2W\sum_i w_i \left(y_i - \hat{y}(t_i | \vec{\theta})\right)
\end{eqnarray}

For shorthand, we will hereafter refer to $M(t_i|\vec{\theta}, \omega)$ as $M_i$.

We will also write things in terms of matrices, vectors, and inner-products:

\begin{equation}
\iprod{x}{y} \equiv \sum_{i=1}^N x_i y_i
\end{equation}

The inner product will always denote a summation over all observations. The
template $M$ can now be expressed as

\begin{equation}
M(t) = A^T\Xc + B^T\Xs
\end{equation}

where $\Xc = \left[\cos{\omega t}, \cos{2\omega t}, ..., \cos{H\omega t} \right]$ and
$\Xs = \left[\sin{\omega t}, \sin{2\omega t}, ..., \sin{H\omega t}\right]$.

\begin{equation}
\partial_bM(t) = \dA^T\Xc + \dB^T\Xs
\end{equation}

Where $\dA$ and $\dB$ are appropriate applications of

\begin{equation}
g'_n(b|p, q) \equiv \partial_b g_n(b|p, q) = n\left(pU_{n-1}(b) \pm q\frac{T_n(b)}{\sqrt{1 - b^2}}\right).
\end{equation}

We can now rewrite the system of equations in terms of inner products:

\begin{eqnarray}
0 &=& \partial_{a} \chi^2 = 2W\left(\iprod{w}{yM} - \iprod{w}{\hat{y}M}\right)\\
0 &=& \partial_{b} \chi^2 = 2aW\left(\iprod{w}{y\dM} - \iprod{w}{\hat{y}\dM}\right)\\
0 &=& \partial_{c} \chi^2 = 2W\left(\iprod{w}{y} - \iprod{w}{\hat{y}}\right)
\end{eqnarray}

The next set of vectors, matrices, and scalars are pre-computed sums that will be solved efficiently with (non-equispaced) fast Fourier transforms:

\begin{eqnarray}
\Cc   &\equiv& \iprod{w}{\Xc}\\
\Ss   &\equiv& \iprod{w}{\Xs}\\
\CCc &\equiv& \iprod{w}{\oprod{\Xc}{\Xc}}\\
\SSs &\equiv& \iprod{w}{\oprod{\Xs}{\Xs}}\\
\CS &\equiv&  \iprod{w}{\oprod{\Xc}{\Xs}}\\
\YC  &\equiv& \iprod{w}{y\Xc}\\
\YS  &\equiv& \iprod{w}{y\Xs}\\
\Ybar &\equiv& \iprod{w}{y}\\
\YY &\equiv& \iprod{w}{y^2}
\end{eqnarray}

We will assume that $\Ybar = 0$, by computing $\Ybar$ from observations
and then substituting $y_i - \Ybar$ for $y_i$.

\subsection{first eqn}

\begin{align}
0 &= \partial_{a} \chi^2 = 2W\left(\iprod{w}{yM} - \iprod{w}{\hat{y}M}\right)\\
\iprod{w}{yM} &= \iprod{w}{\hat{y}M} \\
\iprod{w}{y(A^T\Xc + B^T\Xs)} &= \iprod{w}{(aM + c)M} \\
A^T\iprod{w}{y\Xc} + B^T\iprod{w}{y\Xs} &= a\iprod{w}{M^2} + c\iprod{w}{M}\\
A^T\YC + B^T \YS &= a\iprod{w}{A^T\oprod{\Xc}{\Xc}A + 2A^T\oprod{\Xc}{\Xs}B + B^T\oprod{\Xs}{\Xs}} + c\iprod{w}{A^T\Xc + B^T\Xs}\\
&= a\big(A^T\iprod{w}{\oprod{\Xc}{\Xc}}A + 2A^T\iprod{w}{\oprod{\Xc}{\Xs}}B \\
&\qquad\quad + B^T\iprod{w}{\oprod{\Xs}{\Xs}}B\big) + c\big(A^T\iprod{w}{\Xc} + B^T\iprod{w}{\Xs}\big)\\
&= a(A^T\CCc A + 2A^T\CS B + B^T\SSs B) + c(A^T\Cc + B^T\Ss)
\end{align}

\subsection{2nd eqn}

\begin{align}
0 &= \partial_{b} \chi^2 = 2W(\iprod{w}{y\dM} - \iprod{w}{\hat{y}\dM})\\
\iprod{w}{y\dM} &= \iprod{w}{\hat{y}\dM}\\
\iprod{w}{y\dA^T\Xc} + \iprod{w}{y\dB^T\Xs} &= a\iprod{w}{M\dM} + c\iprod{w}{\dM}\\
\dA^T\YC + \dB^T\YS &= a\big(\iprod{w}{A^T\oprod{\Xc}{\Xc}\dA} + \iprod{w}{A^T\oprod{\Xc}{\Xs}\dB} \\
					&\qquad\quad + \iprod{w}{B^T\oprod{\Xs}{\Xc}\dA} + \iprod{w}{B^T\oprod{\Xs}{\Xs}\dB}\big)\\
					&\qquad\quad + c\big(\iprod{w}{\dA^T\Xc} + \iprod{w}{\dB^T\Xs}\big)\\
					&= a\big(A^T\CCc \dA + A^T\CS\dB + \dA^T\CS B + B^T\SSs \dB\big) \\
					&\qquad\quad + c\big(\dA^T\Cc + \dB^T\Ss\big)
\end{align}

\subsection{3rd eqn}

\begin{align}
0 &= \partial_{c} \chi^2 = 2W(\iprod{w}{y} - \iprod{w}{\hat{y}})\\
\underbrace{\iprod{w}{y}}_{\Ybar=0} &= a\iprod{w}{M} + c\underbrace{\iprod{w}{1}}_{W=1}\\
c &= -a\left(A^T\Cc + B^T\Ss\right)
\end{align}


\subsection{combining eq. 1 and 3}
\begin{align}
A^T\YC + B^T \YS &= a(A^T\CCc A + 2A^T\CS B + B^T\SSs B) + c(A^T\Cc + B^T\Ss)\\
&= a(A^T\CCc A + 2A^T\CS B + B^T\SSs B) - a(A^T\Cc + B^T\Ss)^2\\
&= a\left(A^T\CCc A + 2A^T\CS B + B^T\SSs B - \left(A^T\oprod{\Cc}{\Cc}A + 2A^T\oprod{\Cc}{\Ss}B + B^T\oprod{\Ss}{\Ss}B\right)\right)\\
&= a\left(A^T\underbrace{(\CCc - \oprod{\Cc}{\Cc})}_{\equiv \hatCC}A + 2A^T\underbrace{(\CS - \oprod{\Cc}{\Ss})}_{\equiv \hatCS}B + B^T\underbrace{(\SSs - \oprod{\Ss}{\Ss})}_{\equiv\hatSS}B\right)\\
a &= \frac{A^T\YC + B^T \YS}{A^T\hatCC A + 2A^T\hatCS B + B^T\hatSS B}
\end{align}

\subsection{combining eq. 2 and 3}
\begin{align}
\dA^T\YC + \dB^T\YS &= a\big(A^T\CCc \dA + A^T\CS\dB + \dA^T\CS B + B^T\SSs\dB\big) \\
					&\qquad\quad + c\big(\dA^T\Cc + \dB^T\Ss\big)\\
					&= a\bigg(A^T\CCc \dA + A^T\CS\dB + \dA^T\CS B + B^T\SSs\dB\\
					&\qquad\quad - \left(A^T\Cc + B^T\Ss\right)\left(\dA^T\Cc + \dB^T\Ss\right)\bigg)\\
					&=a\bigg(A^T\CCc \dA + A^T\CS\dB + \dA^T\CS B + B^T\SSs \dB\\
					&\qquad\quad - \left(A^T\oprod{\Cc}{\Cc}\dA + A^T\oprod{\Cc}{\Ss}\dB + \dA^T\oprod{\Cc}{\Ss}B + B^T\oprod{\Ss}{\Ss}\dB \right)\bigg)\\
					&=a\left(A^T\hatCC \dA + A^T\hatCS\dB + \dA^T\hatCS B + B^T\hatSS \dB\right)\\
			   a    &= \frac{\dA^T\YC + \dB^T\YS}{A^T\hatCC \dA + A^T\hatCS\dB + \dA^T\hatCS B + B^T\hatSS \dB}
\end{align}

\subsection{Solutions as roots of a polynomial in $b$}
The final expression depending only on a single free parameter ($b$) is then

\begin{align}
a = \frac{\dA^T\YC + \dB^T\YS}{A^T\hatCC \dA + A^T\hatCS\dB + \dA^T\hatCS B + B^T\hatSS \dB} &= \frac{A^T\YC + B^T \YS}{A^T\hatCC A + 2A^T\hatCS B + B^T\hatSS B}\\
\left(\dA^T\YC + \dB^T\YS\right)\left(A^T\hatCC A + 2A^T\hatCS B + B^T\hatSS B\right) &= \left(A^T\YC + B^T \YS\right)\\
																					&\qquad\quad \times \left(A^T\hatCC \dA + A^T\hatCS\dB\right. \\
																					&\qquad\quad\quad + \left.\dA^T\hatCS B + B^T\hatSS \dB\right)
\end{align}

After simplification, this becomes

\begin{align*}
0 &= A^T\left(\YC\left(A^T\hatCC + B^T\hatCS^T\right) - \hatCC^TA\YC^T\right)\dA\\
  &\quad + B^T\left(\YS\left(A^T\hatCC + B^T\hatCS^T\right) - \left(2\hatCS^T A + \hatSS^T B\right)\YC^T\right)\dA\\
  &\quad + A^T\left(\YC\left(A^T\hatCS + B^T\hatSS\right) - \hatCC^T A \YS^T\right)\dB\\
  &\quad + B^T\left(\YS\left(A^T\hatCS + B^T\hatSS\right) - \left(2\hatCS^T A + \hatSS^T B\right)\YS^T\right)\dB
\end{align*}

Ultimately, this expression should reduce to a single polynomial in $b$, since all components of $A$ and $B$ are polynomials in $b$, ignoring factors of $\sqrt{1 - b^2}$ which can be removed in the final expression with some simple algebra. The optimization of the parameters, then, reduces to finding the zeros of an order $\sim 3H - 1$ polynomial, which can be done numerically.

At this point, I haven't derived the coefficients of this polynomial, but for now we can treat the final expression as a black-box function and numerically find the zeros (albeit probably much more slowly than if we knew the polynomial coefficients).

The final expression of the periodogram has yet to be derived.

\subsubsection{derivation of polynomial}

\begin{align*}
0 &= A_i\YC^iA_j\hatCC^j_k\dA^k + A_i\YC^iB_j(\hatCS^T)^j_k\dA^k - A_i\hatCC^i_j A^j\YC_k\dA^k\\
  &\quad + B_i\YS^iA_j\hatCC^j_k\dA^k + B_i\YS^iB_j(\hatCS^T)^j_k\dA^k - 2B_i(\hatCS^T)^i_jA^j\YC_k\dA^k - B_i(\hatSS^T)^i_jB^j\YC_k\dA^k\\
  &\quad + A_i\YC^iA_j\hatCS^j_k\dB^k + A_i\YC^iB_j\hatSS^j_k\dB^k - A_i(\hatCC^T)^i_j A^j\YS_k\dB^k\\
  &\quad + B_i\YS^iA_j\hatCS^j_k\dB^k + B_i\YS^iB_j\hatSS^j_k\dB^k - 2B_i(\hatCS^T)^i_jA^j\YS_k\dB^k - B_i(\hatSS^T)^i_jB^j\YS_k\dB^k\\
0 &= A_iA_j\dA_k\YC^i\hatCC^{jk} + A_iB_j\dA_k\YC^i\hatCS^{kj} - A_iA_j\dA_k\YC^k\hatCC^{ij}\\
  &\quad + B_iA_j\dA_k\YS^i\hatCC^{jk} + B_iB_j\dA_k\YS^i\hatCS^{kj} - 2B_iA_j\dA_k\hatCS^{ji}\YC^k - B_iB_j\dA_k\hatSS^{ij}\YC^k\\
  &\quad + A_iA_j\dB_k\YC^i\hatCS^{jk} + A_iB_j\dB_k\YC^i\hatSS^{jk} - A_iA_j\dB_k\hatCC^{ij}\YS^k\\
  &\quad + B_iA_j\dB_k\YS^i\hatCS^{jk} + B_iB_j\dB_k\YS^i\hatSS^{jk} - 2B_iA_j\dB_k\hatCS^{ji}\YS^k - B_iB_j\dB_k\hatSS^{ij}\YS^k\\
0 &= \underbrace{A_iA_j\dA_k\left(\YC^i\hatCC^{jk} -  \YC^k\hatCC^{ij}\right)}_{0} + A_iA_j\dB_k\left(\YC^i\hatCS^{jk} - \YS^k\hatCC^{ij}\right)\\
  &\quad + A_iB_j\dA_k\left(\YC^i\hatCS^{kj} + \YS^j\hatCC^{ik}\right) + A_iB_j\dB_k\left( \YC^i\hatSS^{jk} + \YS^j\hatCS^{ik}\right)\\
  &\quad + B_iB_j\dA_k\left(\YS^i\hatCS^{kj} - \YC^k\hatSS^{ij}\right) + \underbrace{B_iB_j\dB_k\left(\YS^i\hatSS^{jk} - \YS^k\hatSS^{ij}\right)}_{0}\\
0 &= A_iA_j\dB_k\left(\YC^i\hatCS^{jk} - \YS^k\hatCC^{ij}\right) + A_iB_j\dA_k\left(\YC^i\hatCS^{kj} + \YS^j\hatCC^{ik}\right)\\
  &\quad + A_iB_j\dB_k\left( \YC^i\hatSS^{jk} + \YS^j\hatCS^{ik}\right) + B_iB_j\dA_k\left(\YS^i\hatCS^{kj} - \YC^k\hatSS^{ij}\right)
%0 &= A^T\left(\YC\left(A^T\hatCC + B^T\hatCS^T\right) - \hatCC^TA\YC^T\right)\dA\\
%  &\quad + B^T\left(\YS\left(A^T\hatCC + B^T\hatCS^T\right) - \left(2\hatCS^T A + \hatSS^T B\right)\YC^T\right)\dA\\
%  &\quad + A^T\left(\YC\left(A^T\hatCS + B^T\hatSS\right) - \hatCC^T A \YS^T\right)\dB\\
%  &\quad + B^T\left(\YS\left(A^T\hatCS + B^T\hatSS\right) - \left(2\hatCS^T A + \hatSS^T B\right)\YS^T\right)\dB
\end{align*}

Let's simplify things a bit with polynomial algebra for polynomials on the interval $[-1, 1]$;

\begin{eqnarray}
p(x) &=& p_0 + p_1x + p_2x^2 + ... + p_nx^n\\
     &=& \vec{p}^T \vec{x}_n
\end{eqnarray}

Where we define $\vec{x}_n = [ x^0, x^1, ..., x^n ]$ and $\vec{p} = [ p_0, p_1, ..., p_n ]$.

Define a binary function $\star$ to denote the product of two polynomials:

\begin{eqnarray}
(p\star q)(x) &=& p_0\left(\vec{q}^T\vec{x}_m\right)x^0 + p_1\left(\vec{q}^T\vec{x}_m\right)x^1 + ... + p_n\left(\vec{q}^T\vec{x}_m\right)x^n\\
	          &=& \vec{pq}^T\vec{x}_{n+m}
\end{eqnarray}

The $A$, $B$, $\dA$, and $\dB$ vectors can be expressed as

\begin{equation}
P_n^{\alpha} = p_n^{\alpha}
\end{equation}

\section{Obtaining an expression for the periodogram}

\begin{align*}
p(\omega) &\equiv 1 - \chi^2_{\rm opt}(\omega) / \chi^2_0
\end{align*}

Next we obtain an expression for $\chi^2_0$:
\begin{align*}
\partial_c \chi^2_0 &= -2W\iprod{w}{y - \hat{y}}\\
	      \underbrace{\iprod{w}{y}}_{=\Ybar=0} &= \underbrace{\iprod{w}{\hat{y}=c}}_{=c}\\
	        c = \hat{y} &= 0\\
	        \chi^2_0 &= W\iprod{w}{y^2}\\
	        \chi^2_0 &= \YY
\end{align*}

Then obtain an expression for $\chi^2_{\rm opt}(\omega)$, recalling that:

\begin{equation}
c = -a\iprod{w}{M}
\end{equation}

and

\begin{equation}
a = \frac{\iprod{w}{yM}}{\iprod{w}{M^2} - \iprod{w}{M}^2}
\end{equation}

\begin{align*}
\chi^2_{\rm opt}(\omega) &= W\iprod{w}{(y - aM - c)^2}\\
&= \iprod{w}{y^2 - 2y(aM + c) + (aM + c)^2}\\
&= \iprod{w}{y^2 - 2ayM - 2cy + a^2M^2 + 2acM + c^2}\\
&= \iprod{w}{y^2} - 2a\iprod{w}{yM} - 2c\underbrace{\iprod{w}{y}}_{=0} + a^2\iprod{w}{M^2} + 2ac\iprod{w}{M} + c^2\underbrace{\iprod{w}{1}}_{=1}\\
&= \YY - 2a\iprod{w}{yM} + a^2\iprod{w}{M^2} + 2ac\iprod{w}{M} + c^2\\
&= \YY - 2a\iprod{w}{yM} + a^2\iprod{w}{M^2} - 2a^2\iprod{w}{M}^2 + a^2\iprod{w}{M}^2\\
&= \YY - 2a\iprod{w}{yM} + a^2\left(\iprod{w}{M^2} - \iprod{w}{M}^2\right)\\
&= \YY - \left(\frac{\iprod{w}{yM}^2}{\iprod{w}{M^2} - \iprod{w}{M}^2}\right)
\end{align*}

From this we obtain the final expression for the periodogram:

\begin{eqnarray}
p(\omega) &=& 1 - \frac{\chi^2_{\rm opt}(\omega)}{\chi^2_0}\\
		&=& 1 - \frac{1}{\YY}\left(\YY - \left(\frac{\iprod{w}{yM}^2}{\iprod{w}{M^2} - \iprod{w}{M}^2}\right)\right)\\
		&=& \frac{1}{\YY}\left(\frac{\iprod{w}{yM}^2}{\iprod{w}{M^2} - \iprod{w}{M}^2}\right)\\
		&=& \frac{1}{\YY}\left(\frac{ A^T(\oprod{\YC}{\YC})A + 2A^T(\oprod{\YC}{\YS})B + B^T(\oprod{\YS}{\YS})B}{ A^T(\CCc - \oprod{\Cc}{\Cc}) A + 2A^T(\CS - \oprod{\Cc}{\Ss}) B + B^T(\SSs - \oprod{\Ss}{\Ss}) B }\right)
\end{eqnarray}


\subsection{Does this reduce to the original LSP when $H=1$?}
When $H = 1$, everything becomes scalar-valued and we can set $\hatCS$ to zero (see Zechmeister \& Kurster) by setting $t_i \rightarrow t_i - \tau_0$, where $\tau_0$ is implicitly defined as

\begin{equation}
\tan2\omega\tau_0 = \frac{2\hatCS}{\hatCC - \hatSS}.
\end{equation}

The $\Xc$ and $\Xs$ then become $\Xct\equiv\cos{\omega(t - \tau_0)}$ and $\Xst\equiv\sin{\omega(t-\tau_0)}$, and all sums involving inner and outer products of $X_s$ and $X_c$ are redefined with respect to this $\tau_0$. We can then set $c_n = c_1 = 1$ and $s_n = s_1 = 0$, and thus $A\rightarrow b$ and $B\rightarrow\pm\sqrt{1 - b^2}$.

We can then solve the equation to obtain $b$:

\begin{align*}
0 &= A^T\left(\YC\left(A^T\hatCC + B^T\hatCS^T\right) - \hatCC^TA\YC^T\right)\dA\\
  &\quad + B^T\left(\YS\left(A^T\hatCC + B^T\hatCS^T\right) - \left(2\hatCS^T A + \hatSS^T B\right)\YC^T\right)\dA\\
  &\quad + A^T\left(\YC\left(A^T\hatCS + B^T\hatSS\right) - \hatCC^T A \YS^T\right)\dB\\
  &\quad + B^T\left(\YS\left(A^T\hatCS + B^T\hatSS\right) - \left(2\hatCS^T A + \hatSS^T B\right)\YS^T\right)\dB\\
  &= A\dA\left(A\left(\YC\times\hatCC\right) - A\left(\YC\times\hatCC\right)\right) + B\dA\left(A\left(\YS\times\hatCC\right) - B\left(\YC\times\hatSS\right)\right) \\
  &\quad + A\dB\left(B\left(\YC\times\hatSS\right) - A\left(\YS\times\hatCC\right)\right) + B\dB\left(B\left(\YS\times\hatSS\right) - B\left(\YS\times\hatSS\right)\right)\\
  &= B\dA\left(A\left(\YS\times\hatCC\right) - B\left(\YC\times\hatSS\right)\right)+ A\dB\left(B\YC\times\hatSS - A\left(\YS\times\hatCC\right)\right)\\
  &= AB\dA\left(\YS\times\hatCC\right) - B^2\dA\left(\YC\times\hatSS\right) + AB\dB\left(\YC\times\hatSS\right) - A^2\dB\left(\YS\times\hatCC\right)\\
  &= \left(b\times\left(\pm\sqrt{1-b^2}\right)\times1\right)\left(\YS\times\hatCC\right) - \left(\left(1-b^2\right)\times 1\right)\left(\YC\times\hatSS\right) \\
  &\quad + \left(b\times\pm\sqrt{1-b^2}\times\left( \mp \frac{b}{\sqrt{1-b^2}}\right)\right)\left(\YC\times\hatSS\right) - b^2\times\left(\mp \frac{b}{\sqrt{1-b^2}}\right)\left(\YS\times\hatCC\right)\\
  &= \pm\left(b\sqrt{1 - b^2}\right)\left(\YS\times\hatCC\right) - \left(1 - b^2\right)\left(\YC\times\hatSS\right)\\
  &\quad - b^2\left(\YC\times\hatSS\right) \pm \left(\frac{b^3}{\sqrt{1-b^2}}\right)\left(\YS\times\hatCC\right)\\
  &= \pm b\left(\sqrt{1 - b^2} + \frac{b^2}{\sqrt{1-b^2}}\right)\left(\YS\times\hatCC\right) - \left(\YC\times\hatSS\right)\\
  &= \pm \left(\frac{b}{\sqrt{1 - b^2}}\right)\left(\YS\times\hatCC\right) - \left(\YC\times\hatSS\right)\\
(1 - b^2)\left(\YC\times\hatSS\right)^2  &= b^2\left(\YS\times\hatCC\right)^2
\end{align*}

And we obtain

\begin{eqnarray}
A = b &=& \pm\frac{|\YC\times\hatSS|}{\sqrt{\left(\YC\times\hatSS\right)^2 +\left(\YS\times\hatCC\right)^2}}\\
B = \pm\sqrt{1 - b^2} &=& \pm\frac{|\YS\times\hatCC|}{\sqrt{\left(\YC\times\hatSS\right)^2 +\left(\YS\times\hatCC\right)^2}}.
\end{eqnarray}

Plugging these values into the expression for the periodogram, we find


\begin{eqnarray}
p(\omega) &=& \frac{1}{\YY}\left(\frac{ A^2(\YC)^2 + 2AB(\YC)(\YS) + B^2(\YS)^2}{ A^2(\hatCC) + B^2(\hatSS)}\right)\\
		  &=& \frac{1}{\YY}\left(\frac{ \left(\frac{A}{B}\right)^2(\YC)^2 + 2\frac{A}{B}(\YC)(\YS) + (\YS)^2}{ \left(\frac{A}{B}\right)^2(\hatCC) + (\hatSS)}\right)\\
		  &=& \frac{1}{\YY}\left(\frac{ \left|\frac{\YC\times\hatSS}{\YS\times\hatCC}\right|^2(\YC)^2 \pm 2\left|\frac{\YC\times\hatSS}{\YS\times\hatCC}\right|(\YC\times\YS) + (\YS)^2}{ \left|\frac{\YC\times\hatSS}{\YS\times\hatCC}\right|^2(\hatCC) + (\hatSS)}\right)\\
		  &=& \frac{1}{\YY}\left(\frac{ \YC^4\times(\hatSS)^2 \pm 2\left(\YC\times\YS\right)^2\times|\hatSS\times\hatCC| + \YS^4\times(\hatCC)^2}{ \YC^2\times\hatSS^2\times\hatCC + \YS^2\times\hatCC^2\times\hatSS}\right)\\
		  &=& \frac{1}{\YY\times\hatCC\times\hatSS}\left(\frac{ \YC^4\times(\hatSS)^2 \pm 2\left(\YC\YS\right)^2\times\left|\hatSS\times\hatCC\right| + \YS^4\times(\hatCC)^2}{ \YC^2\times\hatSS + \YS^2\times\hatCC}\right)\\
		  &=& \frac{1}{\YY\times\hatCC\times\hatSS}\left(\frac{ \left(\YC^2\times\left|\hatSS\right| \pm \YS^2\times\left|\hatCC\right|\right)^2}{ \YC^2\times\hatSS + \YS^2\times\hatCC}\right)
\end{eqnarray}

The $\pm$ ambiguity can be resolved here; we want the \emph{maximal} value of $p(\omega)$, and thus we replace $\pm$ with $+$:

\begin{eqnarray}
p(\omega) &=& \frac{1}{\YY\times\hatCC\times\hatSS}\left(\frac{ \left(\YC^2\times\left|\hatSS\right| + \YS^2\times\left|\hatCC\right|\right)^2}{ \YC^2\times\hatSS + \YS^2\times\hatCC}\right)
\end{eqnarray}

Since 

\begin{eqnarray}
\hatSS &=& \iprod{w}{\left(\sin{\omega t} - \iprod{w}{\sin{\omega t}}\right)^2} = {\rm Var}(\sin{\omega t_i}) > 0\\
\hatCC &=& \iprod{w}{\left(\cos{\omega t} - \iprod{w}{\cos{\omega t}}\right)^2} = {\rm Var}(\cos{\omega t_i}) > 0,
\end{eqnarray}

the periodogram reduces to the original GLS:

\begin{eqnarray}
p(\omega) &=& \frac{1}{\YY}\left(\frac{\YC^2}{\hatCC} + \frac{\YS^2}{\hatSS}\right).
\end{eqnarray}

%\section{Implementation}

%\subsection{Computing coefficients of polynomial}

%\subsection{Root finding}

%Using \texttt{numpy.roots} uses about $\sim 10^{-5}$ seconds per frequency.

%\subsection{Summations using NFFT}


\end{document}